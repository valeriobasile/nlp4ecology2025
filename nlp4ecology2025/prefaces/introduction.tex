We are pleased to welcome you to NLP4Ecology 2025, the 1st International Workshop on Ecology, Environment, and Natural Language Processing. This first edition debuts in a hybrid format on March 2nd, 2025, co-located with the Joint 25th Nordic Conference on Computational Linguistics and the 11th Baltic Conference on Human Language Technologies (NoDaLiDa/Baltic-HLT 2025) in Tallinn, Estonia.

The NLP4Ecology workshop provides a venue for publication and exchange between the Natural Language Processing (NLP) community and stakeholders from different disciplines. It aims to explore how computational linguistics and NLP tools, methods, and applications can contribute to addressing urgent environmental challenges—not only climate change, which has received the most research attention so far due to its visibility and quantifiability, but also broader ecological crises affecting humans, non-human animals, and ecosystems worldwide. Tackling these issues requires interdisciplinary action, and the NLP community has a crucial role to play. The responsibility to address these challenges extends beyond scientists directly involved in climate and environmental studies—it is a shared duty.

NLP4Ecology aims to attract a highly interdisciplinary audience, welcoming contributions at the intersection of linguistics, ecology, and computer science, especially from fields such as AI, computational linguistics, digital humanities, ecolinguistics, ethics, philosophy, and environmental humanities. Our goal is to expand research and collaboration and to empower the NLP community to take an active role in addressing the ecological crisis through innovative and collective action.

This year's program includes a keynote lecture and three presentation sessions. We received 21 submissions, including 10 long papers, 7 short papers, and 4 research communications. Our Program Committee (PC) consisted of 11 early-career and 5 senior researchers, each responsible for reviewing two to three papers. Every submission was double-blind reviewed by two PC members, and we carefully considered their assessments in making our final selection. The PC members did an outstanding job, and we sincerely thank them for their invaluable contributions to maintaining a high-quality program. In the end, we accepted 15 papers: 7 long papers, 6 short papers, and 2 research communications (the latter not included in the proceedings). These numbers yield an overall acceptance rate of 71.4%.

In organizing this hybrid workshop, we sought to preserve as much as possible the engagement and interaction of a fully in-person event. The program includes 7 oral presentations in the opening and closing sessions, while the poster session in between features 9 poster presentations.

Topic-wise, the workshop includes contributions on the development and evaluation of NLP models for ecological and environmental applications, including Green NLP efforts and mitigating anthropocentric biases in large language models. Several papers focus on corpus creation, entity recognition for environmental concepts, and information extraction in the contexts of biodiversity, climate change, and sustainability. Other studies investigate topic modeling and discourse analysis of environmental narratives, covering online discussions, policy documents, and scientific texts. We also feature work on sentiment and emotion analysis in ecological discourse, multilingual approaches to environmental communication, recommender systems for renewable energy communities, and NLP methods for environmental monitoring and advocacy.

Regarding language diversity, the accepted papers explore datasets and experiments involving 6 languages, including English, Finnish, Portuguese (Brazilian), Russian, Spanish, and Italian. By discussing environmental issues from a multilingual perspective, the workshop provides a valuable opportunity to highlight cultural differences in how ecological challenges are framed and addressed across languages.

A workshop of this scale requires the advice, support, and enthusiastic participation of many individuals, to whom we express our deepest gratitude. We especially thank our keynote speaker, Tommaso Caselli (University of Groningen), for his inspiring talk on "Climate Crises, Vegan Meat, and Sustainable Fuels: How Words Shape Reality". We also extend our appreciation to the Program Committee members for their time and dedication in shaping an excellent technical program. Finally, we thank all the authors and participants for making the first edition of NLP4Ecology a success and contributing to the growth of research at the intersection of NLP, ecology, and environmental discourse.

The NLP4Ecology Program Chairs
(in Alphabetical Order)

Valerio Basile, University of Turin, Italy 
Cristina Bosco, University of Turin, Italy 
Francesca Grasso, University of Turin, Italy
Muhammad Okky Ibrohim, University of Turin, Italy
Maria Skeppstedt, Uppsala University, Sweden
Manfred Stede, Potsdam University, Germany
