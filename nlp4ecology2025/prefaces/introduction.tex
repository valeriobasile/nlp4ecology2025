We are pleased to welcome you to NLP4Ecology 2025, the 1st International Workshop on Ecology, Environment, and Natural Language Processing. This first edition debuts in a hybrid format on March 2nd, 2025, co-located with the Joint 25th Nordic Conference on Computational Linguistics and the 11th Baltic Conference on Human Language Technologies (NoDaLiDa/Baltic-HLT 2025) in Tallinn, Estonia.

The NLP4Ecology workshop serves as a venue for publication and exchange between the Natural Language Processing (NLP) community and stakeholders from different disciplines. The aim is to explore how computational linguistics and NLP tools, methods, and applications can help address the pressing challenges of the environment going also beyond the issues related to climate change which attracted most of the research community interest (probably due to the higher visibility and quantifiability of its effects). The anthropogenic ecological crisis, affecting humans, non-human animals, and ecosystems worldwide, poses urgent challenges that require interdisciplinary action. The NLP community has a crucial role to play in tackling these environmental challenges because the responsibility extends beyond scientists directly involved in climate and environmental studies—it is a shared duty.

NLP4Ecology aims to attract a highly interdisciplinary audience and welcomes contributions at the intersection of linguistics, ecology, and computer science, especially from fields such as AI, computational linguistics, digital humanities, ecolinguistics, ethics, philosophy, and environmental humanities. Our goal is to expand research and collaboration and empower the NLP community to take an active role in addressing the ecological crisis through innovative and collective action.

This year's program includes a keynote lecture and three (?) presentation sessions. We received 21 submissions, including 10 long papers, 7 short papers and 4 research communications. For the first time, our Program Committee (PC) included 11 early-career and 5 senior researchers, each responsible for reviewing two to three papers. Each submission was assigned to two PC members and received two reviews. We carefully considered the reviewers’ assessments when making our selection. The PC members did an excellent job in reviewing the submissions, and we sincerely thank them for their essential role in selecting the papers to accept and contributing to a high-quality program. Our goal was to create a balanced program that included as many highly rated papers as possible. At the end, we accepted 16 papers: 7 long papers, 6 short papers, and 3 research communications (the latter not included in the proceedings). These numbers yield an overall acceptance rate of XX%.

In organizing the hybrid workshop, we aimed to preserve as much as possible the experience of a fully in-person conference. A total of 7 oral presentations were given in the opening and closing sessions, while the poster session, in between, included 9 poster presentations.

Topic-wise, we have contribuitions on the development and evaluation of NLP models for ecological and environmental applications, including Green NLP efforts and addressing anthropocentric biases in large language models. Several papers focus on corpus creation, entity related to the environment recognition, and information extraction in the context of biodiversity, climate change, and sustainability. We also feature studies on topic modeling and discourse analysis of environmental narratives, covering online discussions, policy documents, and scientific texts. Other papers explore sentiment and emotion analysis in ecological discourse, multilingual approaches to environmental communication, recommender systems for communities advocating renewable energy communities, and methods for leveraging NLP for environmental monitoring and promotion. As far as the languages involved, the papers discuss data collections and experiments on XXXX different languages, among which English, Italian, Brazilian Portuguese, XXXXX. By discussing in the workshop the environmental issues in a multilingual perspective, we offer important opportunities for opening the NLP research towards cultural differences underlying the different languages.

A workshop of this scale requires the advice, support, and enthusiastic participation of many people, to whom we express our deepest gratitude. We especially thank our keynote speaker, Tommaso Caselli (University of Groningen), for his inspiring talk on "XXX." We also thank the members of our Program Committee for spending their time and dedication in helping us design an excellent technical program. Finally, we thank all the authors who submitted their work and all participants for making the first edition of NLP4Ecology a success and contributing to the growth of research at the intersection of NLP, ecology, and environmental discourse.
