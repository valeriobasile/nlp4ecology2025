We are excited to welcome you to NLP4Ecology 2025, the 1st International Workshop on Ecology, Environment, and Natural Language Processing. This first edition debuts in a hybrid format on March 2nd, 2025, co-located with the Joint 25th Nordic Conference on Computational Linguistics and the 11th Baltic Conference on Human Language Technologies (NoDaLiDa/Baltic-HLT 2025) and held in Tallinn, Estonia.

The NLP4Ecology workshop serves as a venue for publication and exchange between the Natural Language Processing (NLP) community and stakeholders from various disciplines, aiming to explore how computational linguistics and NLP tools, methods, and applications can help address pressing climate change and environment-related challenges. The anthropogenic ecological crisis, which affects people, non-human animals, and ecosystems worldwide, presents urgent challenges that require interdisciplinary action. The NLP community has a crucial role to play in tackling these environmental issues, as the responsibility extends beyond scholars directly involved in climate and environmental studies—it is a shared duty.

NLP4Ecology aims to attract a highly interdisciplinary audience and welcomes contributions at the intersection of linguistics, ecology, and computer science, particularly from fields such as AI, computational linguistics, digital humanities, ecolinguistics, ethics, philosophy, and environmental humanities. Our goal is to expand research and collaboration, empowering the NLP community to take an active role in addressing the ecological crisis through innovative and collective action.

This year, the program includes a keynote talk and three (?) presentation sessions. We received 21 submissions, consisting of 10 long papers, 7 short papers, and 4 research communications (the latter not included in this proceedings volume). For the first time, our Program Committee (PC) included 11 early-career researchers and 5 senior members, each responsible for reviewing two to three papers. Every submission was assigned to two PC members and received two reviews. In making our selections, we carefully considered the reviewers’ assessments. The PC members did an excellent job in reviewing the submissions, and we sincerely thank them for their essential role in selecting the accepted papers and contributing to a high-quality program. Our goal was to create a balanced program that included as many highly rated papers as possible. We accepted 16 papers: 7 long papers, 6 short papers, and 3 research communications (the latter not included in this proceedings volume). These numbers yield an overall acceptance rate of XX%.

In organizing the hybrid workshop, we aimed to preserve as much as possible the experience of a fully in-person conference. The opening and closing sessions featured 7 oral presentations in total, while the poster session, placed between them, included 9 poster presentations.

Topic-wise, we have papers on the development and evaluation of NLP models for ecological and environmental applications, including efforts towards Green NLP and addressing anthropocentric bias in large language models. Several contributions focus on corpus creation, entity recognition, and information extraction related to biodiversity, climate change, and sustainability. We also feature studies on topic modeling and discourse analysis of environmental narratives, covering online discussions, policy documents, and scientific texts. Additionally, papers explore sentiment and emotion analysis in ecological discourse, multilingual approaches to environmental communication, recommender systems for renewable energy communities, and methodologies for leveraging NLP in environmental monitoring and advocacy.

A workshop of this scale requires advice, support, and the enthusiastic participation of many individuals, and we extend our deepest gratitude to all of them. We thank our keynote speaker, Tommaso Caselli (University of Groningen), for his inspiring talk on "XXX." We also express our appreciation to our Program Committee members for their time and dedication in helping us shape an excellent technical program. Finally, we thank all the authors who submitted their work and all the participants for making the first edition of NLP4Ecology a success and for contributing to the growth of research at the intersection of NLP, ecology, and environmental discourse.
