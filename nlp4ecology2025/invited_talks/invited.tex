The words we choose, the contexts in which they are uttered, and the actors conveying these utterances all have a critical role in the way we create narratives that influence our perception of reality. In some cases these narratives can be so strong that they defy the appearance: referring to the increase of the earth’s temperature as “global warming” hinders the destructive effects that higher temperatures have on the climate and the livability of the planet. “Vegan meat” is an oxymoron but the anchoring between the established concept (animal-based food) and the new one (plant-based alternative) can favor the acceptance of the latter in the protein transition debate. When presenting “sustainable fuels” as green solutions for mobility, companies fail to specifiy that sustaianbility is just a reduction of CO2 production when compared to fossil fuels. This talk is focused on the linguistic devices and their use to convey narratives where NLP is a methodology to uncover how the use of words affect the perceptions of different issues related to the ecological transition.